\documentclass[conference]{IEEEtran}
\IEEEoverridecommandlockouts
% The preceding line is only needed to identify funding in the first footnote. If that is unneeded, please comment it out.
\usepackage{cite}
\usepackage{amsmath,amssymb,amsfonts}
\usepackage{algorithmic}
\usepackage{graphicx}
\usepackage{float}
\usepackage{textcomp}
\usepackage{xcolor}
\def\BibTeX{{\rm B\kern-.05em{\sc i\kern-.025em b}\kern-.08em
    T\kern-.1667em\lower.7ex\hbox{E}\kern-.125emX}}

%Para introduzir figura s, utilizar
%\begin{figure}[H]
%	\centering
%	\includegraphics[scale=0.3]{./imagens/[nomedoarquivo]}
%	\caption{Legenda.}
%	\medskip
%	\small
%\end{figure}

\begin{document}

\title{FUNCIONAMENTO E USO SEGURO DE EMA - INCUBADORA NEONATAL}

\author{\IEEEauthorblockN{1\textsuperscript{st} Held, Leonardo J.}
\IEEEauthorblockA{\textit{Universidade Federal de Santa Catarina / Fundação CERTI} \\
\textit{Departamento de Engenharia Elétrica / Centro de Convergência Digital e Mecatrônica }\\
Florianópolis, Brazil \\
leonardo.held@grad.ufsc.br}
\and
\IEEEauthorblockN{2\textsuperscript{nd} Given Name Surname}
\IEEEauthorblockA{\textit{dept. name of organization (of Aff.)} \\
\textit{name of organization (of Aff.)}\\
City, Country \\
email address or ORCID}
\and
\IEEEauthorblockN{3\textsuperscript{rd} Given Name Surname}
\IEEEauthorblockA{\textit{dept. name of organization (of Aff.)} \\
\textit{name of organization (of Aff.)}\\
City, Country \\
email address or ORCID}
}

\maketitle

\begin{abstract}

\end{abstract}

\begin{IEEEkeywords}

\end{IEEEkeywords}

\section{Introdução}

\section{Descrição e Princípio Funcional do \textit{EMA}}
\subsection{Breve descrição não-técnica e histórico}
A incubadora neonatal é um equipamento que proporciona as condições de umidade e temperatura ideias dentro de uma câmara, e é geralmente utilizado em centros de terapia intensiva dentro de uma unidade hospitalar, afim de manter a temperatura corporal de pacientes (infantis) que não conseguem regular a temperatura corporal de maneira apropriada.

O equipamento pode utilizar técnicas de controle em malha fechada e sensoriamento de temperatura e umidade afim de determinar os parâmetros ótimos para uma condição estável do paciente.

Equipamentos do tipo existem desde o século 17, quando se desenvolveram em Paris \textit{designs} baseados em incubadoras de galinhas, afim de manter infantis quentes.

Após a Segunda Guerra Mundial e principalmente nos anos 60, várias tecnologias de suporte ao aparelho respiratório foram desenvolvidas, atribuídas à prosperidade econômica e o boom do pós-guerra, com melhorias e inclusão total do tratamento de doenças congenitais sendo realizado nos hospitais, incluindo a instalação e uso de incubadoras.

Para mais informações na história da incubadora, sugere-se a leitura de \cite{baker}.

Vale-se notar que o público alvo da incubadora neonatal são bebês prematuros e com doenças congenitais em um ambiente clínico. Apesar, incubadoras também são utilizadas com animais domésticos neonatais e para o transporte de pacientes, esta última configuração exigindo especificações diferenciadas, dado a natureza complexa de locomoção de pacientes enfermos.

\subsection{Princípio Funcional}
O funcional da incubadora é simples comparado a outros equipamentos clínicos modernos: o objetivo é manter temperatura e umidade numa certa faixa alvo. Para tal, podem ser utilizados sistemas em malha aberta ou malha fechada, com a variável de controle podendo ser a temperatura do ar (sistema tipo TAC), ou a pele do infatil (sistema tipo TIC, via epidermial).

Uma ventoinha elétrica estilo \textit{fan} sopra ar filtrado sobre uma resistência elétrica e um reservatório de água. 

O ar pode ser umidificado por difusão ou utilizando um evaporador ultrasônico e é esquentado por condução ao encostar nos elementos quentes da resistência. Incubadoras também possuem válvulas de entrada controlada de oxigênio.

Esse ar entra na \textit{cúpula} por uma corrente de convecção e é realimentado pelo sistema ou sai para o ambiente por saídas convencionais ou por compartimentos específicos, que passam por alguns componentes elétricos e ajudam no controle da quantidade de CO2 na incubadora.

\subsection{Estrutura}

A incubadora é formada por três blocos distintos: a base, a cúpula e uma seção intermediária. Uma breve descrição destes três blocos segue.

\begin{itemize}
    \item Base: permite que ajustes ergonomicos como altura sejam realizados, geralmente por controles utilizando os pé.
    \item Cúpula: formada por acrílico transparente, de camada simples ou dupla, com portinholas de acesso e guarnições siliconizadas. Portinholas projetadas pra abertura com o cotovelo, evitando uso das mãos. Também devem permitir a passagem de Raios X.

    Podem ter aberturas tipo íris, afim de passar tubos e fios, com regulagem de abertura e fechamento.

    \item Compartimento Intermediário: Contém base metálica que isola a cúpula e dá posicionamento à bandeja que contém o bebê.

    Abaixo, os sistemas de circulação de ar (filtro), umidificação (com unidade para inserir água no sistema) e controle, com entrada de oxigênio via válvula limitadora.
    
\end{itemize}

\subsection{Informações sobre componentes e alguns parâmetros}

Os filtros utilizados geralmente são do tipo HEPA, filtrando até 99\% de 3 microns\cite{fisher}. 

Observou-se em sites de especializados de venda de insumos para o setor médico, com o \textit{boothmed}, uma gama de potências para os aquecedores, de 200W até 600W.

O método de controle também varia. Equipamentos podem ter medição simples e controle estilo "\textit{on-off}" ou com controladores proporcionais-derivadores-integradores, obtendo resposta de diferentes sensores de temperatura e possivelmente com sensor na pele do paciente (o que a literatura indica ser evitado devido a dificuldade introduzida no manuseio e possíveis danos ao paciente).

As portinholas da incubadora possuem um design particularmente interessante, que deve possibilitar a abertura com o cotovelo, minimizando o contato com superfícies e possivelmente introduzindo patógenos dentro da cúpula.

\section{Estudo de Mercado}

As fabricantes de incubadoras neonatais são diversas, marcas incluem Comen, Drager, Heal Force, Fanem, Olidef dentre outras, sendo Fanem e Olidef as marcas com maior penetração no mercado brasileiro.

O custo de incubadoras é volátil e variável. Fatores alterantes incluem a marca, materiais de fabricação, especificações técnicas e tecnológicas etc.

O valor das incubadoras se encontra entre R\$25.000,00 até R\$45.000,00, com algumas \textit{outliers}, podendo alcançar valores bem mais elevados. O preço, no entanto é compatível com a tecnologia utilizada na fabricação, logicamente dependendo do propósito e qualidade do equipamento.

Na próxima seção, serão abordadas as normas aplicáveis. No entanto, vale-se já falar que todas as incubadoras devem estar apropriadamente registradas junto à ANVISA, no enquadramento de equipamento eletromédico. Isso é um dado importante e é aqui citado pois deve ser verificado na hora da compra. Na nossa consulta, nem todas as lojas e fabricantes colocam à mostra a informação com o número de cadastro do equipamento, podendo acarretar em problemas burocráticos ou até de segurança, em casos extremamente graves.

\section{Normas Aplicáveis}
Além das normas usuais de segurança de equipamentos eletromédicos, como a \texttt{NBR IEC 60601-1}, existem a \texttt{NBR IEC 60601-2-19}, que trata especificamente de prescrições de segurança para incubadoras neonatais e a \texttt{NBR IEC 60601-2-20} e sua errata, que tratam especificamente de incubadoras de transporte. 

O conjunto de padrões específico da \texttt{NBR IEC 60601-2-20} trata sobre critérios de posicionamento de sensores, estipulando que a medida de temperatura da mesa deve ser, por exemplo, feita em cinco pontos distintos.

Além, dois registros vigentes no SOMASUS existem para a incubadora: \texttt{E528 - Incubadora Neonatal (estacionária)} e \texttt{E529 - Incubadora de Transporte Neonatal}. 

Uma leitura complementar de avaliação feita nessas bases normativas pode ser realizada a partir de \cite{rosane}. Também recomenda-se fortemente a leitura das normativas e cadastro no SOMASUS.

\section{Aplicação do EMA na saúde}
\subsection{Profissionais Responsáveis}
Treinamento no uso do equipamento é necessário a todos que trabalham na seção clínica de uma UTI neonatal. A manutenção, como encher o reservatório de água, por exemplo, geralmente recaí sob os cuidados de enfermeiras.

Seguindo os protocolos clínicos conhecidos informalmente como "caos controlado", é importante que quaisquer profissional que entre em contato com o equipamento reconheça sua função, controles, sinalizações e avisos.

Além disso, é de responsabilidade do Engenheiro Clínico fazer conhecimento do estado do equipamento, sua instalação elétrica e seus parâmetros funcionais, observando, por exemplo, se o filtro HEPA foi trocado adequadamente.

\section{Sobre a Plataforma TMH Digital}
\subsection{O conteúdo da plataforma é de fácil entendimento?}
O grupo julgou positivamente o entendimento dos conteúdos vistos no website.

\subsection{Os  equipamentos  apresentados  na  plataforma  precisam  ser atualizados?}
Exposto o nosso trabalho, realizado com fontes externas além do TMH, julgou-se que o TMH fez uma boa introdução ao equipamento médico. Uma atualização que o grupo julgou pertinente, e que auxiliou no compreendimento do conteúdo em questão, foi a busca por vídeos explicativos e didáticos, e de situações reais do uso da incubadora.

\section{Conclusão}
Nesse trabalho, foram observadas características funcionais, normativas e de uso da incubadora, além de um breve resgate histórico da tecnologia. Apesar do foco tecnológico dado no documento, a real relevância são nas inúmeras vidas salvas por essa tecnologia todos os anos. 

O compreendimento técnico aliado com esse nobre propósito permite obter uma sensação de dever e de enorme responsabilidade ao projetar e realizar manutenção em equipamentos dessa natureza, o que contribuí fortemente para a boa formação do Engenheiro Clínico e Biomédico.

\bibliographystyle{IEEEtran}

\bibliography{refs.bib}



\end{document}
